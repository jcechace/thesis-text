\chapter{Úvod}
Bakalářské a diplomové práce představují tradiční součást akademické sféry. Nejspíše každý z~nás s~nimi má zkušenost jako student, někteří jako zadavatelé či vedoucí. Ne každý si však uvědomuje složitost agendy související s~jejich správou a množstvím subjektů, které se celého procesu účastní, obzvláště v~případě, kdy je zahrnuto více univerzit.


Cílem této práce je vytvoření informačního systému pro správu diplomových a bakalářských prací pro společnost Red Hat, průmyslové partnera Fakulty informatiky Masarykovy univerzity. Systém nahradí stávající agendu správy závěrečných prací prostřednictvím firemní wiki stránky, která je s~rostoucím počtem vypisovaných témat nedostačující. Hlavním přínosem nového systému oproti stávající agendě by mělo být značné zjednodušení a zpřehlednění distribuce vypisovaných témat jednotlivým univerzitám.


První část práce se věnuje vysvětlení základních pojmů a představení použitých technologií. V následující části je čtenář blíže seznámen s rámcem Grails, architekturou a vývojem aplikací na této platformě. Poslední dvě části popisují vybrané implementační detaily práce se soubory ve vyvíjeném systému, jeho nasazení a budoucnost
