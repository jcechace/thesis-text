\chapter{Implementace správy souborů}

\section{Archivace souborů}
Jelikož žádný z~dostupných modulů pro práci s~Mongo GridFS nevyhovoval ani základním stanoveným požadavkům, vyvinul jsem jako součást systému pro správu bakalářských prací s~využitím získaných poznatků rozhraní vlastní -- pracovně nazvané MongoFile. Aktuálně se jedná o~součást vyvinutého informačního systému, avšak implementace umístěná v~balíku \texttt{com.redhat.grails.mongodb} počítá s~budoucím osamostatnění projektu v~podobně zásuvného modulu platformy Grails.

\subsection{Servisní třída MongoFile}
Hlavní část implementace je tvořena servisní třídou \texttt{GridFileService} poskytující rozhraní pro manipulaci se soubory. Při návrhu MongoFile jsem zohlednil dva typy souborů:
\begin{itemize}
\item Soubory vázané na doménovou třídu
\item Soubory bez doménových vazeb
\end{itemize}
Pro rozlišení mezi těmito soubory využívá navržené rozhraní možnost definovat několik kořenových kyblíčku GridFS v~rámci jedné databáze. Soubory jsou ukládány do kyblíčku odvozeného z~názvu vázané doménové třídy, případně lze toto jméno změnit definováním statického atributu
\\\texttt{bucketMapping} v~dané doménové třídě. Soubory bez doménové vazby jsou ukládány do kyblíčku výchozího, nazvaného \texttt{fs}. Vazba na konkrétní instanci doménové třídy je tvořena pomocí následujících atributů v~souborových metadatech:
\begin{itemize}
\item \texttt{id} slouží jako identifikátor vázané entity.
\item \texttt{group} klasifikuje soubory v~rámci jedné doménové třídy.
\end{itemize}
Jako identifikátor může sloužit hodnota libovolného atributu vázané entity. Ve výchozím nastavení je použita hodnota atributu \texttt{id}.

\begin{example}
\centering
\begin{lstlisting}

{
    "metadata" : { "group" : "avatar", "id" : NumberLong(6) }
}
\end{lstlisting}
\caption{metadata souboru v~kolekci \texttt{user.files}}
\end{example}


\subsubsection{\textbf{Uložení souboru do databáze}}
Rozhraní MongoFile poskytuje několik metod pro uložení souboru do databáze. Základem je metoda \texttt{save} obsahující hlavní logiku vkládání nového souboru. Při volání této metody je potřeba uvést následující parametry:
\begin{enumerate}
\item \texttt{content}\\
Obsah ukládaného souboru. Zdrojem dat může být instance třídy \texttt{File}, \texttt{MultipartFile}, \texttt{InputStream} nebo pole bytů.
\item \texttt{id}\\
Identifikátor vázané entity nebo \texttt{null} v~případě souboru bez vazby.
\item \texttt{bucket}\\
Kořenový kyblíček použitého GridFS. V~případě hodnoty \texttt{null} se použije výchozí kyblíček.
\item \texttt{group}\\
Zařazení souboru do skupiny.
\item \texttt{metadata}\\
Doplňující informace o~souboru.
\item \texttt{filename}\\
Jméno ukládaného souboru. Pokud je to možné, je jako výchozí hodnota použito jméno původního souboru.
\item \texttt{delete}\\
Způsob jakým postupovat v~případě již existujícího souboru se stejným \texttt{mongoId} nebo kombinací atributů \texttt{id} a \texttt{group}.
\\V případě hodnoty \texttt{true} budou existující soubory smazány.
\item \texttt{mongoId}\\
Unikátní identifikátor dokumentu uloženého v~kolekci \texttt{files}.
\end{enumerate}

Z~množství parametrů metody \texttt{save} je zřejmé, že její použití není příliš pohodlné. Metoda navíc umožňuje libovolnou kombinaci hodnot a uživatele nijak nenutí dodržovat stanovené konvence pro jména kyblíčků či nastavení metadat. Z~tohoto důvodu obsahuje třída \texttt{GridFileService} také přetíženou variantu \texttt{save} umožňující použití jmenných parametrů v~kombinaci s~výchozími hodnotami.

\begin{example}
\centering
\begin{lstlisting}[language=Java]
def thesis = Thesis.get(1)
def file = new File('/home/jcechace/file.zip')

// ulozeni souboru bez domenove vazby
gridFileService.save(file, null, null, null, null, "standalone_file.zip", false, null)

// s pouzitim jmenych atributu
gridFileService.save(content: file, filename: "standalone_file.zip")


// ulozeni souboru s domenovou vazbou
gridFileService.save(file, 1, "thesis", "archives", null, "bound_file.zip", false, null)

// s pouzitim jmenych atributu
gridFileService.save(content: file, filename: "bound_file", object: thesis)
\end{lstlisting}
\caption{uložení souboru do databáze}
\end{example}

\subsubsection{\textbf{Získání souboru z~databáze}}
Rozhraní MongoFile umožňuje uložené soubory z~GridFS získat buďto na základě informací uložených v~metadatech, nebo přímo pomocí \texttt{\_id} dokumentu v~kolekci \texttt{files}.

\begin{example}
\centering
\begin{lstlisting}[language=Java]
def thesis = Thesis.get(1)

// ziskani souboru z kyblicku "thesis" podle _id dokumentu
def file = gridFileService.getFileByUniversalMongoId(mongoId, "thesis")

// ziskani souboru z kyblicku podle metadat
def file = gridFileService.getFile(['id': 1, group: "archives"], "thesis")

// misto konkreniho kyblicku lze uvest domenovou tridu
def file = gridFileService.getFile(['id': 1, group: "archives"], Thesis.class)
\end{lstlisting}
\caption{získání souboru z~databáze}
\end{example}

V~případě vyhledávání souboru na základě metadat vrátí volání metody \texttt{getAllFiles} pouze jednu instanci \texttt{GridFSDBFile} reprezentující soubor s~nejnižším \texttt{\_id}. Pro získání všech vyhovujících souborů jsou k~dispozici přetížené varianty vracející seznam objektů typu \texttt{GridFSDBFile}.

\subsection{Servírování souboru klientovi}
Poslední, avšak neméně důležitou, schopností MongoFile je zaslání získaného souboru klientské aplikaci. Metoda \texttt{serveFile} třídy \texttt{GridFileService} nastaví potřebné hlavičky, jako například \texttt{content-type}, a zapíše požadovaný soubor do výstupního proudu HTTP odpovědi.
\begin{example}
\centering
\begin{lstlisting}[language=Java]
// ziskani souboru z databaze
def file = gridFileService.getFile(['id': 1, group: "archives"], Thesis.class)

// zapsani souboru do http odpovedi
gridFileService.serveFile(response, file, true)
\end{lstlisting}
\caption{servírování souboru klientovi}
\end{example}
Na příkladu lze vidět, že volání metody \texttt{serveFile} vyžaduje tři parametry:
\begin{enumerate}
\item \texttt{Response}: objekt reprezentující odpovědi HTTP protokolu.
\item \texttt{File}: instanci třídy \texttt{GridFSDBFile} zasílanou klientovi.
\item \texttt{Save}: vynucení stažení souboru webovým prohlížečem.
\end{enumerate}
Pravděpodobně nejčastější použití této metody bude jako součást akce v~controlleru. Proto jsem v~rámci implementace MongoFile vytvořil jednoduchý controller \texttt{GridFileController}, schopný při volání akce \texttt{serveFile} zaslat soubor s~požadovaným identifikátorem.

\subsubsection{\textbf{Knihovna značek pro práci s~GridFile}}
Pro usnadnění přístupu k~souborům z~prezentační vrstvy jsem implementoval také knihovnu značek \texttt{GridFileTagLib}. Knihovna využívá jmenný prostor \texttt{grid} a obsahuje modifikované verze značek \texttt{link} a \texttt{createLink} ze standardní knihovny Grails. První značka slouží pro vytvoření hypertextového odkazu na soubor, druhá pak pouze pro sestavení adresy využitelné například v~html značce \texttt{<img>}.

\section{Nahrávání souborů}
Implementace funkcionality pro nahrávání souborů se skládá ze dvou částí. Klienta umožňujícího zaslat soubory na server, který následně interpretuje výsledek provedené operace v~uživateli přívětivé podobně. Právě tuto část řeší použití nástroje Fine Uploader\cite{fu-homepage}. Implementací serverové části, zajišťující zpracování přijatých dat, a její spolupráci s~klientskou části aplikace se zabývají následující podkapitoly.

\subsection{Přijetí a zpracování nahrávaných dat}
Celý proces nahrání souboru od uživatele na server lze rozdělit do 4 kroků
\begin{enumerate}
\item Uživatel vybere soubor.
\item Klientská aplikace zašle data na server.
\item Server příjme zaslaná data, zpracuje je a zašle zpět odpověď.
\item Uživateli je zobrazen výsledek.
\end{enumerate}

Jak bylo zmíněno na začátku kapitoly, Fine Uploader se z~vetší části postará o~všechny body s~výjimkou čísla tři. Příjem a zpracování dat se při použití architektury MVC běžně provádí vytvořením akce v~příslušném controlleru. V~informačních systémech je však nezřídka potřeba zpracovávat soubory určené pro různé části systému různým způsobem. Definování akce  pro každý druh souboru však často povede k~množství opakujícího se kódu.

\begin{example}
\centering
\begin{lstlisting}[language=Java]
def uploadToMongo() {
    try {
        MultipartFile file = ((MultipartHttpServletRequest) request).getFile('qqfile')

        // ulozeni souboru
        def saved = gridFileService.save(file,object: springSecurityService.currentUser, group: 'files')

        // vraceni vysledku
        return render(text: [success: true] as JSON, contentType: 'text/json')
    } catch (Exception e) {
        // nastala chyba
        log.error("Error during file upload", e)
        def errorMessage = message(code: 'uploader.error.upload')

        return render(text: [success: false, message: errorMessage] as JSON, contentType: 'text/json')
    }
}
\end{lstlisting}
\caption{akce pro zpracování přijatého souboru}
\end{example}

Na příkladu výše lze vidět, že samotné uložení souboru do databáze tvoří jen minimální část celé akce. Zbytek kódu se proto bude nejspíš opakovat ve většině podobných metod. V~systému pro správu diplomových prací minimalizuji množství takového kódu pomocí reaktivního přístupu.

\subsubsection{\textbf{Příjem souborů}}
V~rámci vyvinutého systému jsem vytvořil controller \texttt{com.redhat.grails\\.upload.UploadController}, jehož akce \texttt{upload} slouží jako jediný koncový bod pro příjem souborů v~celém systému. Místo zpracování přijatého souboru však tato akce za použití servisní třídy \texttt{UploadService} v~systému vyvolá událost obsahující data potřebná ke zpracování. Konkrétní vyvolané události lze specifikovat pomocí parametru \texttt{topic} zaslaného jako součást HTTP požadavku spolu se souborem. Logika pro zpracování přijatého souboru je potom součástí metody reagující na vyvolanou událost. Systém pro správu diplomových prací používá události \texttt{avatar} a \texttt{thesis}. První slouží k~archivaci profilových obrázků uživatelů, druhý k~archivaci souborů asociovaných s~pracemi studentů. Obsluhu těchto událostí lze najít ve třídě \texttt{com.redhat.theses.listeners.UploadListenerService}

\subsubsection{\textbf{Knihovna značek}}
Pro snadnější použití nástroje Fine Uploader a eliminaci redundantního kódu jsem vytvořil knihovnu značek \texttt{UploaderTaglib}, která slouží k~zapouzdření konfigurace a integraci s~platformou Grails -- včetně podpory internacionalizace nástroje.
\begin{example}
\centering
\begin{lstlisting}
<u:uploader topic="thesis" params="${[id: thesisInstance.id]}">
    <u:trigger id="triggerUpload" class="tms-btn">
        <i class="icon-upload"></i>
        <g:message code="uploader.text.startUpload.button"/>
    </u:trigger>
</u:uploader>
\end{lstlisting}
\caption{použití nástroje Fine Uploader pomocí definovaných značek}
\end{example}
